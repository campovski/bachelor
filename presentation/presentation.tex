\documentclass[t, 10pt]{beamer}

\usepackage{amsmath}
\usepackage{amsfonts}
\usepackage[slovene]{babel}
\usepackage[utf8]{inputenc}
\usepackage{lmodern}
\usepackage{tikz}
\usepackage{url}

\newcounter{countDefinicija}
\newenvironment{definicija}{\noindent\refstepcounter{countDefinicija}\textbf{Definicija \arabic{countDefinicija} }\itshape}{\\~}

\newcounter{countLema}
\newenvironment{lema}{\noindent\refstepcounter{countLema}\textbf{Lema \arabic{countLema} }\itshape}{\\~}

\newcounter{countIzrek}
\newenvironment{izrek}{\noindent\refstepcounter{countIzrek}\textbf{Izrek \arabic{countIzrek} }\itshape}{\\~}

\newcounter{countPosledica}
\newenvironment{posledica}{\noindent\refstepcounter{countPosledica}\textbf{Posledica \arabic{countPosledica} }\itshape}{\\~}

\newcommand{\fillblack}[1]{
\begin{tikzpicture}[remember picture, overlay]
    \node [shift={(0 cm,0cm)}]  at (current page.south west)
        {%
        \begin{tikzpicture}[remember picture, overlay] at (current page.south west)
            \draw [fill=black] (0, 0) -- (0,#1 \paperheight) --
                              (\paperwidth,#1 \paperheight) -- (\paperwidth,0) -- cycle ;
        \end{tikzpicture}
        };
        \draw (current page.north west) rectangle (current page.south east);
\end{tikzpicture}
}

%\usetheme{warsaw}


\begin{document}
\title{Algoritem potisni-povišaj za iskanje maksimalnega pretoka}
\author[Marcel Čampa]{Marcel Čampa\\{\small Mentor: Sergio Cabello}}
\institute{Fakulteta za matematiko in fiziko}
\date{12. maj 2017}

\maketitle


\section{Osnovne definicije}

\begin{frame}\frametitle{Osnovne definicije}
\begin{definicija}
\textbf{(Usmerjen) graf} $G$ je par množic $G = (V,E)$, kjer je $V$ množica vozlišč grafa, $E$ pa je množica povezav grafa $G$.
\end{definicija}

\begin{definicija}
Naj bo $G' = (V, E)$ graf. \textbf{Omrežje} $G$ je trojica $G = (V,E, c)$, kjer je $c \colon V \times V \rightarrow \mathbb{R}_+ \cup \{\infty\}$ \textbf{funkcija kapacitivnosti}, ki vsaki povezavi $(u,v)$ priredi njeno kapaciteto $c(u,v)$. Velja $c(u,v) = \infty$ natanko tedaj, ko kapaciteta povezave $(u,v)$ ni omejena.
\end{definicija}

Rekli bomo še, da $c(u,v)=0$ natanko tedaj, ko povezava ne obstaja.

\fillblack{0.33}

\end{frame}

\begin{frame}
\begin{definicija}
Naj bo $G = (V,E,c)$ omrežje. \textbf{Pretočno omrežje} na omrežju $G$ je peterica $(V,E,c,s,t)$, kjer je $s\in V$ začetno vozlišče pretočnega omrežja, rečemo mu \textbf{izvir}, $t\in V$ pa končno vozlišče pretočnega omrežja, ki mu pravimo \textbf{ponor}.
\end{definicija}


\begin{definicija}
\textbf{Psevdopretok} je funkcija $f \colon V \times V \rightarrow \mathbb{R}$, ki zadošča pogojema
\begin{enumerate}
\item Za vsaki vozlišči $u,v \in V$ velja $f(u,v) = - f(v,u)$.
\item Za vsaki vozlišči $u,v \in V$ velja $f(u,v) \leq c(u,v)$, kjer je $c$ funkcija kapacitivnosti.
\end{enumerate}~
\end{definicija}

\fillblack{0.33}
\end{frame}

\begin{frame}
\begin{definicija}
\textbf{Funkcija presežka} za psevdopretok $f$ je funkcija $e_f \colon V \rightarrow \mathbb{R}$, definirana z $e_f(u) = \sum_{v \in V} f(v,u)$. Če je $e_f(u) > 0$, pravimo, da je $u$ \textbf{v presežku}.
\end{definicija}

\begin{definicija}
\textbf{Predpretok} $f$ je tak psevdopretok, v katerem za vsak $v \in V \setminus\{s\}$ velja, da je neto tok, ki priteče v vozlišče $v$, nenegativen, torej da velja $e_f(v) \geq 0$.
\end{definicija}
\fillblack{0.33}
\end{frame}

\begin{frame}
\begin{definicija}
\textbf{Pretok} $f$ je tak psevdopretok, v katerem za vsak $v \in V \setminus\{s,t\}$ velja, da je neto tok, ki priteče v vozlišče $v$, enak nič, torej da velja $e_f(v) = 0$.
\end{definicija}

\begin{definicija}
\textbf{Vrednost pretoka} $f$ označimo z $|f|$ in je definiran kot $|f| = e_f(t)$.
\end{definicija}

\begin{definicija}
\textbf{Maksimalni pretok} je pretok $f$, za katerega velja \[|f| = \max_{f_i} |f_i|.\]
\end{definicija}

\fillblack{0.33}
\end{frame}

\section{Algoritem potisni-povišaj}

\begin{frame}\frametitle{Algoritem potisni-povišaj}
Intuicija...
\fillblack{0.33}
\end{frame}

\begin{frame}[fragile]
\begin{verbatim}
POVIŠAJ (u)
1   // Vozlišče u povišamo, če je e(u) > 0 in
2   // za vsak v iz V, (u,v) v E_f, velja h(u) <= h(v).
3   h(u) = min{h(v) : (u,v) v E_f} + 1
\end{verbatim}

\begin{verbatim}
POTISNI (u, v)
1   // Potisnemo lahko, če je e(u) > 0
2   // c(u,v) > 0 in h(u) = h(v) + 1.
3   delta = min{ e(u), c(u,v) - f(u,v) }
4   f(u,v) += delta
5   f(v,u) -= delta
6   e(u) -= delta
7   e(v) += delta
\end{verbatim}

\fillblack{0.30}
\end{frame}

\begin{frame}[fragile]
\begin{verbatim}
INICIALIZIRAJ_PREDPRETOK(G,s)
 1   // V grafu G si izberemo vozlišče s
 2   // in inicializiramo predtok.
 3   ZA vsak v v V(G)
 4       h(v) = 0
 5       e(v) = 0
 6   ZA vsak (u,v) v E(G)
 7       f(u,v) = 0
 8   h(s) = |V|
 9   ZA vsak v, za katerega obstaja (s,v) v E(G)
10       f(s,v) = c(s,v)
11       e(v) = f(s,v)
\end{verbatim}

\fillblack{0.33}
\end{frame}

\begin{frame}[fragile]
\begin{verbatim}
POTISNI-POVIŠAJ(G,s)
1   INICIALIZIRAJ_PREDPRETOK(G,s)
2   DOKLER obstaja mogoča operacija POTISNI ali POVIŠAJ
3       ČE lahko izvedeš POTISNI, POTEM
4           POTISNI
5       DRUGAČE POVIŠAJ
\end{verbatim}
\fillblack{0.33}
\end{frame}

\section{Pravilnost delovanja algoritma}

\begin{frame}\setcounter{countLema}{0}\frametitle{Pravilnost delovanja algoritma in časovna zahtevnost}

\begin{lema}\label{lem:potisk_ali_povisanje}
Naj bo $G=(V,E,c,s,t)$ pretočno omrežje, $f$ predpretok, $h$ višinska funkcija in $e_f\colon V \rightarrow \mathbb{N}_0$ funkcija presežka. Če ima vozlišče $u\in V$ presežek toka, torej $e_f(u) > 0$, potem lahko na tem vozlišču opravimo ali operacijo potisni ali operacijo povišaj.
\end{lema}
\fillblack{0.33}\pause

\begin{lema}\label{lem:h_ostane_visinska}
Med izvajanjem programa \texttt{POTISNI-POVIŠAJ} $h$ vedno zadrži lastnosti višinske funkcije.
\end{lema}
\fillblack{0.33}
\pause

\begin{lema}\label{lem:ni_poti}
Naj bo $G=(V,E,c,s,t)$ pretočno omrežje, $f$ predpretok v $G$ in $h$ višinska funkcija na $V$. Potem ne obstaja pot od $s$ do $t$ v residualnem omrežju $G_f$.
\end{lema}
\fillblack{0.33}
\end{frame}

\begin{frame}\setcounter{countIzrek}{0}
\begin{izrek}
\textbf{(maksimalni pretok - minimalni prerez)} Naj bo $G=(V,E,c,s,t)$ pretočno omrežje in $f$ pretok. Potem sta naslednji trditvi ekvivalentni:
\begin{enumerate}
\item Pretok $f$ je maksimalni pretok v $G$.
\item Residualno omrežje $G_f$ ne vsebuje povečujoče poti.
\end{enumerate}~
\end{izrek}

\fillblack{0.33}
\pause

\begin{izrek}
Naj bo $G=(V,E,c,s,t)$ pretočno omrežje. Če poženemo algoritem \texttt{POTISNI-POVIŠAJ} na pretočnem omrežju $G$ in se ustavi, potem je predpretok $f$, ki ga algoritem vrne, enak maksimalnemu toku skozi pretočno omrežje $G$.
\end{izrek}
\fillblack{0.33}
\end{frame}

\begin{frame}\setcounter{countLema}{3}\setcounter{countPosledica}{0}

\begin{lema}\label{lem:st_povisanj_vozlisca}
Naj bo $G = (V,E,c,s,t)$ pretočno omrežje. Na koncu izvajanja algoritma \texttt{POTISNI-POVIŠAJ} na $G$, za vsak $u \in V$ velja $h(u) \leq 2|V| - 1$.
\end{lema}
\fillblack{0.33}
\pause

\begin{posledica}\label{pos:om_st_op_povisaj}
Naj bo $G=(V,E,c,s,t)$ pretočno omrežje. Potem je število operacij \texttt{POVIŠAJ} med izvajanjem algoritma \texttt{POTISNI-POVIŠAJ} manjše od $2|V|^2$.
\end{posledica}
\fillblack{0.33}
\pause

\begin{lema}\label{lem:om_st_op_potisni_nas}
Naj bo $G=(V,E,c,s,t)$ pretočno omrežje. Potem je število operacij \texttt{POTISNI}, ki ne zasičijo povezave, med izvajanjem algoritma \texttt{POTISNI-POVIŠAJ} manjše od $2|V||E|$.
\end{lema}
\fillblack{0.33}
\pause

\begin{lema} \label{lem:om_st_op_potisni_nezas}
Naj bo $G=(V,E,c,s,t)$ pretočno omrežje. Potem je število operacij \texttt{POTISNI}, ki ne zasičijo povezave, med izvajanjem algoritma \texttt{POTISNI-POVIŠAJ} manjše od $4|V|^2 (|V| + |E|)$.
\end{lema}
\fillblack{0.33}
\end{frame}

\begin{frame}
\begin{izrek}
Algoritem \texttt{POTISNI-POVIŠAJ} med izvajanjem naredi $\mathcal{O}(V^2E)$ osnovnih operacij.
\end{izrek}
\fillblack{0.33}
\end{frame}


\section{Baseball elimination}

\begin{frame}
\frametitle{Eliminacija ekip v baseballu}
\fillblack{0.33}
\end{frame}

\begin{frame}
\frametitle{Literatura}
\begin{thebibliography}{99}

\bibitem{uvtg}
R.~J.~Wilson, J.~J.~Watkins, \emph{Uvod v teorijo grafov}, Knjižnica Sigma št. 63, DMFA-založništvo, 1997.

\bibitem{clrs}
T.~H.~Cormen, C.~E.~Leiserson, R.~L.~Rivest in C.~Stein, \emph{Introduction to Algorithms}, MIT Press, Massachusetts, 2009.

\bibitem{baseball}
Kevin Wayne, \emph{Baseball Elimination}, dostopno na: \url{https://www.cs.princeton.edu/~wayne/papers/baseball_talk.pdf}, zadnji dostop: 10. maj 2017.

\bibitem{baseball_yt}
UC Davis, \emph{End of Season Elimination: Details}, dostopno na: \url{https://www.youtube.com/watch?v=TiTHIPPatFw}, zadnji dostop: 10. maj 2017.
\end{thebibliography}
\fillblack{0.33}
\end{frame}







































\end{document}
