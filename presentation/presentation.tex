\documentclass[t]{beamer}

\usepackage{amsmath}
\usepackage{amsfonts}
\usepackage[slovene]{babel}
\usepackage[utf8]{inputenc}
\usepackage{lmodern}
\usepackage{tikz}

\newcounter{countDefinicija}
\newenvironment{definicija}{\noindent\refstepcounter{countDefinicija}\textbf{Definicija \arabic{countDefinicija} }\itshape}{\\~}

\newcounter{countTrditev}
\newenvironment{trditev}{\noindent\refstepcounter{countTrditev}\textbf{Trditev \arabic{countTrditev} }\itshape}{\\~}

\newcommand{\fillblack}[1]{
\begin{tikzpicture}[remember picture, overlay]
    \node [shift={(0 cm,0cm)}]  at (current page.south west)
        {%
        \begin{tikzpicture}[remember picture, overlay] at (current page.south west)
            \draw [fill=black] (0, 0) -- (0,#1 \paperheight) --
                              (\paperwidth,#1 \paperheight) -- (\paperwidth,0) -- cycle ;
        \end{tikzpicture}
        };
        \draw (current page.north west) rectangle (current page.south east);
\end{tikzpicture}
}

%\usetheme{warsaw}


\begin{document}
\title{Algoritem potisni-povišaj za iskanje maksimalnega pretoka}
\author{Marcel Čampa}
\institute{Fakulteta za matematiko in fiziko}
\date{\today}

\maketitle


\section{Osnovne definicije}

\begin{frame}
\begin{definicija}
\textbf{Graf} $G$ je par množic $G = (V,E)$, kjer je $G$ množica vozlišč grafa, $E$ pa je množica povezav grafa $G$.
\end{definicija}

\begin{definicija}
Naj bo $G = (V, E)$ graf. \textbf{Omrežje} na grafu $G$ je par $(G, c)$, kjer je $c \colon V \times V \rightarrow \mathbb{R}_+ \cup \{\infty\}$ \textbf{funkcija prepustnosti}, ki vsaki povezavi $(u,v)$ priredi njeno prepustnost $c(u,v)$. Prepustnost $c(u,v) = \infty$ natanko tedaj, ko prepustnost povezave ni omejena.
\end{definicija}

Rekli bomo še, da $c(u,v)=0$ natanko tedaj, ko povezava ne obstaja.

\fillblack{0.33}

\end{frame}

\begin{frame}
\begin{definicija}
Naj bo $G = (V,E)$ graf in $(G,c)$ omrežje na grafu $G$. \textbf{Pretočno omrežje} na omrežju $(G,c)$ je četverica $(G,c,s,t)$, kjer je $s\in V$ začetno vozlišče pretočnega omrežja, rečemo mu \textbf{izvir}, $t\in V$ pa končno vozlišče pretočnega omrežja, ki mu pravimo \textbf{ponor}.
\end{definicija}

Ponavadi pišemo pretočno omrežje kar kot $G = (V,E,s,t)$. Pri tem namreč privzemamo, da imamo neko funkcijo prepustnosti $c$.

\fillblack{0.4}
\end{frame}

\begin{frame}
\begin{definicija}
\textbf{Psevdopretok} je funkcija $f \colon V \times V \rightarrow \mathbb{R}$, ki zadošča pogojema
\begin{enumerate}
\item Za vsaki vozlišči $u,v \in V$ velja $f(u,v) = - f(v,u)$.
\item Za vsaki vozlišči $u,v \in V$ velja $f(u,v) \leq c(u,v)$, kjer je $c$ funkcija prepustnosti.
\end{enumerate}~
\end{definicija}
\begin{definicija}
\textbf{Funkcija presežka} za psevdopretok $f$ je funkcija $e_f \colon V \rightarrow \mathbb{R}$, definirana z $e_f(u) = \sum_{v \in V} f(v,u)$. Če je $e_f(u) > 0$, pravimo, da je $u$ \textbf{v presežku}.
\end{definicija}
\fillblack{0.4}
\end{frame}

\begin{frame}
\begin{definicija}
\textbf{Residualna prepustnost} povezave glede na trenuten psevdopretok $f$ je funkcija $c_f \colon V \times V \rightarrow \mathbb{R}_+$, definirana kot razlika prepustnosti povezave in trenutnega toka preko nje. Velja torej $c_f(u,v) = c(u,v) - f(u,v)$.
\end{definicija}

\begin{definicija}
\textbf{Predpretok} $f$ je tak psevdopretok, v katerem za vsak $v \in V \setminus\{S\}$ velja, da je neto tok, ki priteče v vozlišče $v$, nenegativen, torej da velja $e_f(v) \geq 0$.
\end{definicija}

\fillblack{0.4}
\end{frame}

\begin{frame}
\begin{definicija}
\textbf{Pretok} $f$ je tak psevdopretok, v katerem za vsak $v \in V \setminus\{s,t\}$ velja, da je neto tok, ki priteče v vozlišče $v$, enak nič, torej da velja $e_f(v) = 0$.
\end{definicija}

\begin{definicija}
\textbf{Vrednost pretoka} $f$ je tok, ki vstopa v ponor $t$. Označimo ga z $|f|$. Velja torej $|f| = e_f(t)$.
\end{definicija}

\begin{definicija}
\textbf{Maksimalni pretok} je pretok $f$, za katerega velja \[|f| = \max_{f_i} |f_i|.\]
\end{definicija}

\fillblack{0.4}
\end{frame}











































\end{document}
